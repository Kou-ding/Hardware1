% Document class and two-column conversion
\documentclass{report}
% dimensions of paper and relative text positioning
\usepackage[a4paper,top=2cm,bottom=2cm,left=2cm,right=2cm]{geometry}
% math symbols
\usepackage{amsmath}
\usepackage{amssymb}
% package for including URLs
\usepackage{url}
% Required for including images
\usepackage{graphicx}
\usepackage{float} % Required for specifying the exact location of a figure

\usepackage{minted} % Required for including code snippets

% enable writing in greek
\usepackage[greek,english]{babel}
\usepackage[utf8]{inputenc}

\setlength{\parindent}{0pt} % Removes all indentation from paragraphs

% Start of the document
\begin{document}

% Set the language to greek
\selectlanguage{greek}

% Title page
\title{\Huge \bfseries \selectlanguage{english} Hardware 1 \\ Project 2024\selectlanguage{greek}} %\Huge and \bfseries are used to make the title bigger and bold
\author{Παπαδάκης Κωνσταντίνος Φώτιος\vspace{0.5cm} \\  ΑΕΜ:10371} % \vspace{0.5cm} is used to add some vertical space between the author and the AEM
\date{\today}
% prints the title, author and date on a separate page
\maketitle

\section*{Άσκηση 1}
Στην πρώτη άσκηση μας ζητείται να παράξουμε μια αριθμιτική λογική μονάδα. 
Δημιουργούμε ένα  

\section*{Άσκηση 2}
Για τη δεύτερη άσκηση συνθέτουμε μια αριθμομηχανή η οποία αξιοποιεί την $alu$ της
προηγούμενης άσκησης. Δημιουργούμε δύο αρχεία:
\begin{itemize}
    \item \textbf{$calc_enc\.v:$} Υλοποιεί με $structural verilog$ το σήμα $alu_op$ βάσει 
    της κατάστασης των τριών κουμπιών ($btnr$, $btnc$, $btnl$), 
    \item \textbf{$calc\.v:$} Συγκεντρώνει τη λειτουργικότητα των $alu$, $calc_enc$ και 
    επιπρόσθετα ενημερώνει τον συσσωρευτή βάσει της κατάστασης των κουμπιών $btnu$ και $btnd$.
    Επίσης φροντίζει να αντικατοπτρίζεται ο συσσωρευτής στο $LED$ και δημιουργεί με επέκταση 
    προσήμου δύο $32-bit$ εκδοχές του συσσωρευτή και του διακόπτη για να τις προωθήσει στην
    $alu$ η οποία δέχεται $32-bit$ εισόδους.
\end{itemize}

Κυματομορφές προσομοίωσης:
\begin{figure}[H]
    \centering
    \includegraphics[width=0.5\textwidth]{media/exercise2_waveforms.png}
\end{figure}

\section*{Άσκηση 3}
Στη τρίτη άσκηση υλοποιούμε ένα αρχείο καταχωρητών το οποίο θα αποθηκεύει τις 
τιμές των καταχωρητών του επεξεργαστή $RISK\_V$ πάνω στον οποίο βασίζεται η
εργασία μας. Ο κώδικας είναι αρκετά απλός:
\begin{itemize}
    \item Αρχικοποιούμε τους καταχωρητές με την τιμή 0.
    \item Κάθε φορά που υπάρχει αλλαγή σε κάποια είσοδο του $module$ μας,
    διαβάζουμε τις τιμές που προσδιορίζουν οι $readReg1$ και $readReg2$. 
    Δίνουμε προτεραιότητα στην εγγραφή διαβάζοντας απευθείας από το
    $writeData$ στην περίπτωση που το $write enable$ παίρνει τιμή $1$.
    \item Στην θετική ακμή του ρολογιού, όταν το $write enable$ είναι $1$
    και το $writeReg$ δεν είναι $0$, τότε γράφουμε την τιμή του $writeData$
    στον καταχωρητή που προσδιορίζει το $writeReg$.
\end{itemize}
Για να είμαστε σίγουροι ότι ο κώδικάς μας είναι ορθός φτιάχνουμε ένα $testbench$
το οποίο ελέγχει 5 πράγματα:
\begin{itemize}
    \item Ότι όλοι οι καταχωρητές είναι αρχικοποιημένοι στο 0.
    \item Ότι μπορώ να γράψω μια τιμή σε έναν καταχωρητή και έπειτα να την διαβάσω.
    \item Ότι όταν πάω να διαβάσω την τιμή από τον καταχωρητή 0, διαβάζω την τιμή 0.
    Σε έναν επεξεργαστή $RISK\_V$ ο καταχωρητής 0 είναι πάντα 0.
    \item Ότι όταν πάω να διαβάσω μια τιμή από έναν καταχωρητή όσο το $write$ είναι $1$,
    τότε διαβάζω την τιμή που έγραψα.
    \item Ότι μπορώ να διαβάσω ταυτόχρονα τις τιμές των δύο καταχωρητών.
\end{itemize}


\section*{Άσκηση 4}
Η άσκηση νούμερο τέσσερα περιλαμβάνει την υλοποίηση της δομής $datapath$.
Αξιοποιεί τα προηγούμενο μας $module$:
\begin{itemize}
    \item $alu$
    \item $regfile$
\end{itemize}
όμως τα τεστ υλοποιούνται σε συγχώνευση με τα τεστ του $module$ που υλοποιούμε 
στην άσκηση πέντε $top\_proc$, έτσι ώστε να ελέγξουμε την ολοκληρωμένη λειτουργία
του $RISK\_V$ επεξεργαστή μας.
\\


\section*{Άσκηση 5}
Σχηματικό διάγραμμα από το $FSM$:
% \begin{figure}[H]
%     \centering
%     \includegraphics[width=1\textwidth]{media/exercise5_fsm.png}
% \end{figure}

Κυματομορφές προσομοίωσης:
% \begin{figure}[H]
%     \centering
%     \includegraphics[width=1\textwidth]{media/exercise5_waveforms.png}
% \end{figure}

\section*{Σχόλια}
Τα εργαλεία που μας δόθηκαν για την ανάπτυξη υλισμικού ήταν πολύ περιοριστικά σε 
σχέση με τα εργαλεία της ανάπτυξης λογισμικού που έχουμε συνηθίσει. Το $Quartus$ ήταν ένα 
εξαιρετικά ιδιότροπο πρόγραμμα με μια εξαιρετικά δύσκολη διαδικασία εγκατάστασης και ενεργοποίησης, 
όπου απαιτούνταν από τον μέχρι και το $MAC$ της κάρτας δικτύου. Το $EDA$ $Playground$ είναι ένα $online$ 
εργαλείο το οποίο ήταν εξαιρετικά βολικό στα πλαίσια της εργασίας αλλά για μεγάλα $project$ δεν θα ήταν 
κατάλληλο. Κύρια μειονεκτήματα του ήταν:
\begin{itemize}
    \item Ο $compiler$ του έδινε ασαφή μηνύματα σφαλμάτων και έτσι ήταν δύσκολη η διαδικασία του $debugging$.
    \item Η συνεχής, συχνή αποσύνδεση.
    \item Η αδυναμία επανασύνδεσης έπειτα από αδράνεια του χρήστη.
    \item Η $online$ υπόσταση του, καθιστώντας το ανίκανο να ανταπεξέλθει σε καταστάσεις όπου δεν υπάρχει
    σύνδεση στο διαδίκτυο.
\end{itemize}
Αξίζει να σημειωθεί ότι η βιβλιογραφία που επισυναπτόταν με την εκφώνηση της εργασίας κάλυπτε πλήρως 
τις ανάγκες της εργασίας και ήταν εξαιρετικά χρήσιμη.
\\
\smallskip
\\
Προς το τέλος της εργασίας ανακάλυψα το πακέτο $iverilog$ των $arch$ $linux$ το οποίο συμπεριφέρεται 
ακριβώς όπως ένας $c$ $compiler$ και δούλεψε άψογα με τα $testbenches$ που είχα συνθέσει. Σε συνδυασμό 
με το πακέτο $gtkwave$ μπόρεσα και να οπτικοποιήσω τα αποτελέσματα.
\\
\smallskip
\\

\end{document}