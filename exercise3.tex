Στη τρίτη άσκηση υλοποιούμε ένα αρχείο καταχωρητών το οποίο θα αποθηκεύει τις 
τιμές των καταχωρητών του επεξεργαστή $RISK\_V$ πάνω στον οποίο βασίζεται η
εργασία μας. Ο κώδικας είναι αρκετά απλός:
\begin{itemize}
    \item Αρχικοποιούμε τους καταχωρητές με την τιμή 0.
    \item Κάθε φορά που υπάρχει αλλαγή σε κάποια είσοδο του $module$ μας,
    διαβάζουμε τις τιμές που προσδιορίζουν οι $readReg1$ και $readReg2$. 
    Δίνουμε προτεραιότητα στην εγγραφή διαβάζοντας απευθείας από το
    $writeData$ στην περίπτωση που το $write enable$ παίρνει τιμή $1$.
    \item Στην θετική ακμή του ρολογιού, όταν το $write enable$ είναι $1$
    και το $writeReg$ δεν είναι $0$, τότε γράφουμε την τιμή του $writeData$
    στον καταχωρητή που προσδιορίζει το $writeReg$.
\end{itemize}
Για να είμαστε σίγουροι ότι ο κώδικάς μας είναι ορθός φτιάχνουμε ένα $testbench$
το οποίο ελέγχει 5 πράγματα:
\begin{itemize}
    \item Ότι όλοι οι καταχωρητές είναι αρχικοποιημένοι στο 0.
    \item Ότι μπορώ να γράψω μια τιμή σε έναν καταχωρητή και έπειτα να την διαβάσω.
    \item Ότι όταν πάω να διαβάσω την τιμή από τον καταχωρητή 0, διαβάζω την τιμή 0.
    Σε έναν επεξεργαστή $RISK\_V$ ο καταχωρητής 0 είναι πάντα 0.
    \item Ότι όταν πάω να διαβάσω μια τιμή από έναν καταχωρητή όσο το $write$ είναι $1$,
    τότε διαβάζω την τιμή που έγραψα.
    \item Ότι μπορώ να διαβάσω ταυτόχρονα τις τιμές των δύο καταχωρητών.
\end{itemize}
