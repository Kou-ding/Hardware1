Στην πρώτη άσκηση μας ζητείται να παράξουμε μια αριθμητική λογική μονάδα. 
Δημιουργούμε ένα $module$ το οποίο έχει 3 εισόδους και 2 εξόδους, 9 σταθερές
που δηλώνουν στον πολυπλέκτη ποια πράξη θα γίνει στην $alu$ και μια $switch$ 
η οποία ανάλογα με την τιμή που δέχεται ο πολυπλέκτης επιλέγει την αντίστοιχη
πράξη, όπως προδιαγράφεται στην εκφώνηση.
\\
Επιπρόσθετα φτιάχνουμε και ένα $testbench$ το οποίο ελέγχει όλες τις πιθανές
περιπτώσεις των εισόδων που μπορεί να δεχθεί το συγκεκριμένο module. Με $port
mapping$ περνάμε τις παραμέτρους της $alu$ στο $testbench$, φτιάχνουμε μια
συνάρτηση που να ελέγχει αν ταυτίζεται το αναμενόμενο αποτέλεσμα με το πραγματικό
και παράλληλα κρατάμε στο νου η παράμετρος $zero$ να είναι $1$ όταν το αποτέλεσμα
είναι $0$.